\NeedsTeXFormat{LaTeX2e}
\documentclass[12pt,utf8,a4paper]{article}
\usepackage[utf8]{inputenc}
\usepackage[T1]{fontenc}
\usepackage[pdftex]{graphicx}
\usepackage[ngerman]{babel}

\setlength{\parindent}{0pt}
\setlength{\topmargin}{0pt}
\setlength{\textheight}{26cm}

\newcommand{\respektomat}{{\tt Respektomat}}
\newcommand{\respekt}{{\frqq{}Respekt\flqq{}}}

\begin{document}

\pagestyle{empty}

{\raggedleft
  Jürgen Reuter\\
  \respektomat{} (2019)\\
  Interaktive Objektinstallation\\
  Rechner, Anzeige, Tastatur\\
}
\vspace{2em}

{\center\LARGE{\respektomat}\\}
\vspace{2em}

Der \respektomat{} ist eine interaktive Installation, die den Besucher
zur Eingabe einer Aussage zum Thema \respekt{} auffordert und
respektvoll Stellung auf die eingegebene Aussage bezieht.  Der
\respektomat{} stützt sich für seine Antworten auf eine Sammlung aus
Zitaten aus der deutschsprachigen Wikipedia, in denen das Wort
\respekt{} oder eine sprachliche Ableitung dessen enthalten ist.  Dem
eingegeben Text entsprechend wird aus der Sammlung ein mehr oder
minder passendes Zitat gesucht und zur Anzeige gebracht, oder, wenn
nichts Passendes gefunden wird, eine allgemeine Phrase ausgegeben, die
zur respektvollen Umformulierung der Eingabe auffordert.

Zwar enthält die Wikipedia eigens einen eigenes Lemma zum Begriff
\respekt.  Bemerkenswert jedoch ist der Befund, dass das Wort nur
gelegentlich in den Sachartikeln selbst, dafür aber umso häufiger auf
den Diskussionsseiten der Wikipedia zu finden ist.  Thematisch geht es
dabei meist um eine Diskussion darüber, ob für diesen oder jenen
Artikel ein Löschantrag (im Wikipedia-Jargon \frqq{}LA\flqq{})
gestellt werden solle und inwieweit dabei Respekt dem Autor oder den
übrigen Wikipedianern zu zollen sei.

\end{document}
